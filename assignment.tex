\documentclass{article}
\usepackage{fontenc}
\usepackage{fancyhdr}
\usepackage[utf8]{inputenc}

\title{COS1004 Computer Systems Assignment 2 Part B}
\author{James Hassall, 102100517, COS10004}
\date{23 October 2019}

\pagestyle{fancy}
\lhead{James Hassall (102100517)}
\rhead{COS10004 Computer Systems Assignment 2 Part B}

\begin{document}

\maketitle

\pagebreak
\section*{Introduction}
I am trying to make the classic Rock, Paper, Scissors (RPS) game in assembly. I initially tried to make
a calculator in assembly but realise the time to implement such an application is too long. I 
decide to make RPS becasue I could make it realitively easy, where there are two inputs and 3 outputs
(including the screen). The users makes a selection either Rock, Paper or no button down for Scissors, 
then the computer will make a decision after 4 seconds and and display its answer.  An indicator LED 
will light up red or green depending if the player lost or won respectively.

\section*{Design Outline}
Physical components of the build was 2 10k ohm resistors, 2 1k Ohm resistors, 2 buttons,
2 LED's and a lot of wiring, the physical setup of the system is realitively easy due to the simplicity
of the hardware. The software components was written in FASM becasue that is what I have been taught.
The functions I have created are the 3 different actions Rock, Paper and Scissors where they link up to
a created function called DrawChars (located in DrawChar.asm) which will handle the drawing of the characters
on to the display. I have also created a new fucntion for the Drawing the AI's selection to the screen 
as well as writing the gameplay loop where the ai creates a psuedo random number between 0-2 which is Rock, Paper
or Scissors respectively. Then picks one and its compared to the users output.
\\
The way the 

\section*{Assumptions}
I have assumed that using the GPIO reference doc that was in the earlier labs was fair to use.

\section*{Unresolved Problems}
As of current build the Ai only ever chooses Scissors and therefore the user can always win.

\pagebreak
\section*{Running Program}

\end{document}
